%-------------------------------------------------------------------------------
% This file provides a skeleton ATLAS note.
\pdfinclusioncopyfonts=1
% This command may be needed in order to get \ell in PDF plots to appear. Found in
% https://tex.stackexchange.com/questions/322010/pdflatex-glyph-undefined-symbols-disappear-from-included-pdf
%-------------------------------------------------------------------------------
% Specify where ATLAS LaTeX style files can be found.
\RequirePackage{latex/atlaslatexpath}
% You can comment out the above line if the files are in a central location, e.g. $HOME/texmf.
%-------------------------------------------------------------------------------
\documentclass[NOTE, REPORT=true, atlasdraft=true, UKenglish]{atlasdoc}
% The language of the document must be set: usually UKenglish or USenglish.
% british and american also work!
% Commonly used options:
%  atlasdraft=true|false This document is an ATLAS draft.
%  NOTE                  The document is an ATLAS note (draft).
%  REPORT=true|false     Use scrreprt|scrartcl as main class for document.
%  paper=a4|letter       Set paper size to A4 (default) or letter.

%-------------------------------------------------------------------------------
% Extra packages:
\usepackage{atlaspackage}
% Commonly used options:
%  subfigure|subfig|subcaption  to use one of these packages for figures in figures.
%  minimal               Minimal set of packages.
%  default               Standard set of packages.
%  full                  Full set of packages.
%-------------------------------------------------------------------------------

% Style file with biblatex options for ATLAS documents.
\usepackage{atlasbiblatex}
% Commonly used options:
%  backref=true|false    Turn on or off back references in the bibliography.

% Package for creating list of authors and contributors to the analysis.
\usepackage{atlascontribute}

% Useful macros
\usepackage{atlasphysics}
% See doc/atlas_physics.pdf for a list of the defined symbols.
% Default options are:
%   true:  journal, misc, particle, unit, xref
%   false: BSM, hepparticle, hepprocess, hion, jetetmiss, math, process,
%          other, snippets
% See the package for details on the options.

% Macro to add to-do notes (for several authors). Uses the todonotes package.
% \ATLnote{JS}{Jane}{green!20}{green!50!black!60}
% add macros \JSnote and \JSinote for notes in the margin and inline.
% The first colour is for the body and the second for the border of the note.
% Set output=false in order not to print out the notes.
% Set shift=false to avoid adjustment of margins.
% Check for notes still left by commenting out the package in the final version of the note.
\usepackage[output=true, shift=true]{atlastodo}

% Files with references for use with biblatex.
% Note that biber gives an error if it finds empty bib files.
% \addbibresource{ANA-HION-2025-03-INT1.bib}
\addbibresource{bib/ATLAS.bib}
\addbibresource{bib/CMS.bib}
\addbibresource{bib/ConfNotes.bib}
\addbibresource{bib/PubNotes.bib}

% Paths for figures - do not forget the / at the end of the directory name.
\graphicspath{{logos/}{figures/}}

% Add your own definitions here (file ANA-HION-2025-03-INT1-defs.sty).
\usepackage{ANA-HION-2025-03-INT1-defs}

%-------------------------------------------------------------------------------
% Generic document information.
%-------------------------------------------------------------------------------

% Title, abstract and document.
%-------------------------------------------------------------------------------
% This file contains the title, author and abstract.
% It also contains all relevant document numbers used for an ATLAS note.
%-------------------------------------------------------------------------------

% Title
\AtlasTitle{Charged particle $\RAA$ in $\OO$ and $\NeNe$ collisions at $\sqn=\qty{5.36}{TeV}$}

% Draft version:
% Should be 1.0 for the first circulation, and 2.0 for the second circulation.
% If given, adds draft version on front page, a 'DRAFT' box on top of each other page, 
% and line numbers.
% Comment or remove in final version.
\AtlasVersion{0.3}

% Abstract - % directly after { is important for correct indentation
\AtlasAbstract{The analysis investigates possible suppression or enhancement of inclusive charged hadron production in $\sqn=5.36$~TeV collisions of \OO and \NeNe compared to the \pp collisions at the same energy measured by the ATLAS detector at the LHC. The results are compared to theoretical predictions.}

% Author - this does not work with revtex (add it after \begin{document})
\author{The ATLAS Collaboration}

% Authors and list of contributors to the analysis
% \AtlasAuthorContributor also adds the name to the author list
% Include package latex/atlascontribute to use this
% Use authblk package if there are multiple authors, which is included by latex/atlascontribute
% \usepackage{authblk}
% Use the following 3 lines to have all institutes on one line
% \makeatletter
% \renewcommand\AB@affilsepx{, \protect\Affilfont}
% \makeatother
% \renewcommand\Authands{, } % avoid ``. and'' for last author
% \renewcommand\Affilfont{\itshape\small} % affiliation formatting
% \AtlasAuthorContributor{First AtlasAuthorContributor}{a}{Author's contribution.}
% \AtlasAuthorContributor{Second AtlasAuthorContributor}{b}{Author's contribution.}
% \AtlasAuthorContributor{Third AtlasAuthorContributor}{a}{Author's contribution.}
% \AtlasContributor{Fourth AtlasContributor}{Contribution to the analysis.}
% \author[a]{First Author}
% \author[a]{Second Author}
% \author[b]{Third Author}
% \affil[a]{One Institution}
% \affil[b]{Another Institution}


% If a special author list should be indicated via a link use the following code:
% Include the two lines below if you do not use atlasstyle:
% \usepackage[marginal,hang]{footmisc}
% \setlength{\footnotemargin}{0.5em}
% Use the following lines in all cases:
% \usepackage{authblk}
% \author{The ATLAS Collaboration%
% \thanks{The full author list can be found at:\newline
%   \url{https://atlas.web.cern.ch/Atlas/PUBNOTES/ATL-PHYS-PUB-2016-007/authorlist.pdf}}
% }

% ATLAS reference code, to help ATLAS members to locate the paper
\AtlasRefCode{ANA-HION-2025-03}

% ATLAS note number. Can be an COM, INT, PUB or CONF note
\AtlasNote{ANA-HION-2025-03-INT1}

% Author and title for the PDF file.
\hypersetup{pdftitle={ATLAS document},pdfauthor={The ATLAS Collaboration}}

%-------------------------------------------------------------------------------
% Content
%-------------------------------------------------------------------------------
\begin{document}

\maketitle

\tableofcontents

% List of contributors - print here or after the Bibliography.
% \PrintAtlasContribute{0.30}
% \clearpage

% List of to-do notes.
% \listoftodos

%-------------------------------------------------------------------------------
\chapter{Introduction}
\label{sec:intro}
%-------------------------------------------------------------------------------

Place your introduction here.

%-------------------------------------------------------------------------------
\chapter{Analysis}
\label{sec:analysis}
%-------------------------------------------------------------------------------

You can find some text snippets that can be used in papers in \texttt{latex/atlassnippets.sty}.
To use them, provide the \texttt{snippets} option to \texttt{atlasphysics}.

%-------------------------------------------------------------------------------
\chapter{Results}
\label{sec:result}
%-------------------------------------------------------------------------------

Place your results here.

% All figures and tables should appear before the summary and conclusion.
% The package placeins provides the macro \FloatBarrier to achieve this.
% \FloatBarrier

%-------------------------------------------------------------------------------
\chapter{Conclusion}
\label{sec:conclusion}
%-------------------------------------------------------------------------------

Place your conclusion here.

%-------------------------------------------------------------------------------
% If you use biblatex and either biber or bibtex to process the bibliography
% just say \printbibliography here.
\printbibliography
% If you want to use the traditional BibTeX you need to use the syntax below.
% \bibliographystyle{obsolete/bst/atlasBibStyleWithTitle}
% \bibliography{ANA-HION-2025-03-INT1,bib/ATLAS,bib/CMS,bib/ConfNotes,bib/PubNotes}
%-------------------------------------------------------------------------------

%-------------------------------------------------------------------------------
% Print the list of contributors to the analysis.
% The argument gives the fraction of the text width used for the names.
%-------------------------------------------------------------------------------
\clearpage
The supporting notes for the analysis should also contain a list of contributors.
This information should usually be included in \texttt{mydocument-metadata.tex}.
The list should be printed either here or before the Table of Contents.
\PrintAtlasContribute{0.30}

%-------------------------------------------------------------------------------
\clearpage
\appendix
\part*{Appendices}
\addcontentsline{toc}{part}{Appendices}
%-------------------------------------------------------------------------------

In an ATLAS note, use the appendices to include all the technical details of your work
that are relevant for the ATLAS Collaboration only (e.g.\ dataset details, software release used).
This information should be printed after the Bibliography.

\end{document}
